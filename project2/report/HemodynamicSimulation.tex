\documentclass[11pt]{article} % use larger type; default would be 10pt

% Common packages for mathematical articles
\usepackage{fullpage,amsmath,amsfonts,mathpazo,microtype,nicefrac,algorithm2e,graphicx}

% MSE Packages
\usepackage{caption}
\usepackage{physics}
\usepackage{xcolor}
\usepackage{listings}
\usepackage{parskip}

% MSE Make margins a bit smaller
\usepackage[margin=0.75in]{geometry}
\usepackage[font=footnotesize]{caption}

% MSE set graphics path
\graphicspath{{figs/}}

% Set-up for hypertext references
\usepackage{hyperref,color,textcomp}
\definecolor{webgreen}{rgb}{0,.35,0}
\definecolor{webbrown}{rgb}{.6,0,0}
\definecolor{RoyalBlue}{rgb}{0,0,0.9}
\hypersetup{
   colorlinks=true, linktocpage=true, pdfstartpage=3, pdfstartview=FitV,
   breaklinks=true, pdfpagemode=UseNone, pageanchor=true, pdfpagemode=UseOutlines,
   plainpages=false, bookmarksnumbered, bookmarksopen=true, bookmarksopenlevel=1,
   hypertexnames=true, pdfhighlight=/O,
   urlcolor=webbrown, linkcolor=RoyalBlue, citecolor=webgreen,
   pdfsubject={Harvard IACS AC 290 R},
   pdfkeywords={},
   pdfcreator={pdfLaTeX},
   pdfproducer={LaTeX with hyperref}
}
\hypersetup{pdftitle={AC290-RBC}}


% MSE Macros
\newcommand{\tty}[1]{\texttt{#1}}
\newcommand{\uu}{\vectorbold{u}}
\newcommand{\vv}{\vectorbold{u}}
\newcommand{\Laplace}{\Delta}

%***Start of Document***
\title{Hemodynamic Simulation with Lattice Boltzman}
\author{Harvard IACS AC 290R - Group 1 \\
Michael S. Emanuel \\
Jonathan Guillotte-Blouin \\
Yue Sun \\
}
\date{28-April-2019} 

\begin{document}
\maketitle

\section{Problem Statement and Motivation}
AC 290R is a course on extreme computing with a focus on the application domain of fluid dynamics.
In our first module we attempted a prototypical problem using the continuum description of fluids
governed by the Navier-Stokes equation: 
\href{https://en.wikipedia.org/wiki/Rayleigh%E2%80%93B%C3%A9nard_convection}{Rayleigh-B\'enard Convection}.
In this module, we shift from the continuum to the mesoscale description and simulate fluids using  
\href{https://en.wikipedia.org/wiki/Lattice_Boltzmann_methods}{Lattice Boltzmann} methods.  
These methods are based on the Boltzmann Equation of thermodynamics and statistical physics.

The particular task we we undertook was a hemodynamic simulation.
\href{https://en.wikipedia.org/wiki/Hemodynamics}{Hemodynamics} is the study of the dynamics flow blood.  
It is a rich field at the intersection of anatomy and physics with a storied history.
Pioneers in the field have included 
\href{https://en.wikipedia.org/wiki/Leonardo_da_Vinci}{Leonardo Da Vinci}, 
\href{https://en.wikipedia.org/wiki/Leonhard_Euler}{Leonhard Euler,} 
\href{https://en.wikipedia.org/wiki/Thomas_Young_(scientist)}{Thomas Young}, and 
\href{https://en.wikipedia.org/wiki/Jean_L%C3%A9onard_Marie_Poiseuille}{Jean L.M. Poieseuille}.

The particular problem was as follows.  
We aim to model the dispersion of a therapeutic drug that is injected with a catheter to treat a stenotic artery.
Stenosis is a disease of the arteries in which an artery becomes narrowed.
It is frequently caused by 
\href{https://en.wikipedia.org/wiki/Atherosclerosis}{atherosclerosis}, 
a build-up of fatty deposits (cholesterol) on the walls of the artery.
Stenotic arteries can cause serious medical problems including heart attacks and strokes.
One approach to treating stenoses is to introduce a therapeutic drug with a catheter,
a small tube inserted surgically into the patient that can be threaded through the circular system.
The goal of our simulation was to understand how the drug molecules dispersed 
and to see how many of them passed through the stenotic region, and at what times.

This is a worthwhile object of study for both pedagogical and practical reasons.
Pedagogically, this topic rounds out our survey of both techniques and domain knowledge in the course.
We are aiming to master the techniques of extreme computing and their application to fluid dynamics.
The first topic covered the continuum approach, and this topic covers the mesoscale approach.
The first topic used traditional CPU-centric computations, and this topic introduces us to GPU computing.

On a practical level, \href{https://www.cdc.gov/heartdisease/facts.htm}{heart disease} 
has been for many years a leading cause of death in Americans alongside cancer.
It is a complex disease with many treatment options and a need for accurate diagnosis.
Clinical practice still often relies on human judgment about which arterial blockages 
look risky, and there is reason for optimism that advances in biologically realistic computer
simulations could lead to materially faster and more accurate diagnosis 
and improve treatement selections.

\section{Description of Code}

\section{Overview of Numerical Methods Used}

\section{Parameters of the Simulation}

\section{Results}

\section{Conclusions and Future Work}

%% bibliography

\begin{thebibliography}{9}
\bibitem{exo} 
Succi, Sauro: The Lattice Boltmann Equation for Fluid Dynamics and Beyond

\end{thebibliography}

\end{document}

