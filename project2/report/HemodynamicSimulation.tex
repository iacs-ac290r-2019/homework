\documentclass[11pt]{article} % use larger type; default would be 10pt

% Common packages for mathematical articles
\usepackage{fullpage,amsmath,amsfonts,mathpazo,microtype,nicefrac,algorithm2e,graphicx}

% MSE Packages
\usepackage{caption}
\usepackage{physics}
\usepackage{xcolor}
\usepackage{listings}
\usepackage{parskip}

% MSE Make margins a bit smaller
\usepackage[margin=0.75in]{geometry}
\usepackage[font=footnotesize]{caption}

% MSE set graphics path
\graphicspath{{figs/}}

% Set-up for hypertext references
\usepackage{hyperref,color,textcomp}
\definecolor{webgreen}{rgb}{0,.35,0}
\definecolor{webbrown}{rgb}{.6,0,0}
\definecolor{RoyalBlue}{rgb}{0,0,0.9}
\hypersetup{
   colorlinks=true, linktocpage=true, pdfstartpage=3, pdfstartview=FitV,
   breaklinks=true, pdfpagemode=UseNone, pageanchor=true, pdfpagemode=UseOutlines,
   plainpages=false, bookmarksnumbered, bookmarksopen=true, bookmarksopenlevel=1,
   hypertexnames=true, pdfhighlight=/O,
   urlcolor=webbrown, linkcolor=RoyalBlue, citecolor=webgreen,
   pdfsubject={Harvard IACS AC 290 R},
   pdfkeywords={},
   pdfcreator={pdfLaTeX},
   pdfproducer={LaTeX with hyperref}
}
\hypersetup{pdftitle={AC290-RBC}}


% MSE Macros
\newcommand{\tty}[1]{\texttt{#1}}
\newcommand{\uu}{\vectorbold{u}}
\newcommand{\vv}{\vectorbold{u}}
\newcommand{\Laplace}{\Delta}
\renewcommand{\vec}[1]{\mathbf{#1}}

%***Start of Document***
\title{Hemodynamic Simulation with Lattice Boltzman}
\author{Harvard IACS AC 290R - Group 1 \\
Michael S. Emanuel \\
Jonathan Guillotte-Blouin \\
Yue Sun \\
}
\date{28-April-2019} 

\begin{document}
\maketitle

\section{Problem Statement and Motivation}
AC 290R is a course on extreme computing with a focus on the application domain of fluid dynamics.
In our first module we attempted a prototypical problem using the continuum description of fluids
governed by the Navier-Stokes equation: 
\href{https://en.wikipedia.org/wiki/Rayleigh%E2%80%93B%C3%A9nard_convection}{Rayleigh-B\'enard Convection}.
In this module, we shift from the continuum to the mesoscale description and simulate fluids using  
\href{https://en.wikipedia.org/wiki/Lattice_Boltzmann_methods}{Lattice Boltzmann} methods.  
These methods are based on the Boltzmann Equation of thermodynamics and statistical physics.

The particular task we we undertook was a hemodynamic simulation.
\href{https://en.wikipedia.org/wiki/Hemodynamics}{Hemodynamics} is the study of the dynamics flow blood.  
It is a rich field at the intersection of anatomy and physics with a storied history.
Pioneers in the field have included 
\href{https://en.wikipedia.org/wiki/Leonardo_da_Vinci}{Leonardo Da Vinci}, 
\href{https://en.wikipedia.org/wiki/Leonhard_Euler}{Leonhard Euler,} 
\href{https://en.wikipedia.org/wiki/Thomas_Young_(scientist)}{Thomas Young}, and 
\href{https://en.wikipedia.org/wiki/Jean_L%C3%A9onard_Marie_Poiseuille}{Jean L.M. Poieseuille}.

The particular problem was as follows.  
We aim to model the dispersion of a therapeutic drug that is injected with a catheter to treat a stenotic artery.
Stenosis is a disease of the arteries in which an artery becomes narrowed.
It is frequently caused by 
\href{https://en.wikipedia.org/wiki/Atherosclerosis}{atherosclerosis}, 
a build-up of fatty deposits (cholesterol) on the walls of the artery.
Stenotic arteries can cause serious medical problems including heart attacks and strokes.
One approach to treating stenoses is to introduce a therapeutic drug with a catheter,
a small tube inserted surgically into the patient that can be threaded through the circular system.
The goal of our simulation was to understand how the drug molecules dispersed 
and to see how many of them passed through the stenotic region, and at what times.

This is a worthwhile object of study for both pedagogical and practical reasons.
Pedagogically, this topic rounds out our survey of both techniques and domain knowledge in the course.
We are aiming to master the techniques of extreme computing and their application to fluid dynamics.
The first topic covered the continuum approach, and this topic covers the mesoscale approach.
The first topic used traditional CPU-centric computations, and this topic introduces us to GPU computing.

On a practical level, \href{https://www.cdc.gov/heartdisease/facts.htm}{heart disease} 
has been for many years a leading cause of death in Americans alongside cancer.
It is a complex disease with many treatment options and a need for accurate diagnosis.
Clinical practice still often relies on human judgment about which arterial blockages 
look risky, and there is reason for optimism that advances in biologically realistic computer
simulations could lead to materially faster and more accurate diagnosis 
and improve treatement selections.

\section{Overview of Numerical Methods Used}
The main numerical method used in this simulation is the Lattice Boltzmann Method (LBM).
LBM is based on the \href{https://en.wikipedia.org/wiki/Boltzmann_equation}{Boltzmann Equation},
which describes the statistical behavior of a thermodynamic system and dates to 1872.
The idea behind the Boltzmann Equation is that particles in the system each have 6 degrees
of freedom, 3 positions and 3 momenta along the 3 coordinate directions $x$, $y$ and $z$.
Molecules of a given type are physically indistuinguishable from each other,
so the system can be described completely by the populations of particles as a function
of time and these six dimensions in the phase space.
The Boltzmann Equation describes the evolution of such a system.
The populations of particles change due to three terms: 
forces applied to the system, diffusion, and collisions.

LBM is a technique in computational fluid dynamics that uses the Boltzmann Equation
to devise a numerical simulation of a fluid that can accurately capture mesoscale dynamics
when it is tuned properly.
The physical system is discretized, typically on a rectilinear grid 
and most commonly one with cubic spacing.  
Particle velocities are also discretized, with particles allowed to jump from
from one grid point only to nearby grid points over one simulation step.
The most common discretization scheme for 3D fluid simulations,
which we used for this problem, is called D3Q19.
The label can be parsed as referring to 3 dimensions and 19 discrete velocities.
The 19 discrete velocities have the following structure:
\begin{itemize}
\item 1 $0^{th}$ neighbor; displacement $(0,0,0)$; weight $\frac{1}{3}$
\item 6 $1^{st}$ neighbors; displacement one of 3 permutations of $(\pm 1, 0, 0)$; weight $\frac{1}{18}$
\item 12 $2^{nd}$ neighbors; displacement one of 3 permutations of $\pm 1, \pm 1, 0)$; weight $\frac{1}{36}$
\end{itemize}
The 6 first neighbors have displacements $(1,0,0), (-1,0,0), (0,1,0), (0,-1,0), (0,0,1), (0,0,-1)$.
The 12 second neighbors follow a similar pattern; there are ${3 \choose 2} = 3$ permutations of indices
$i, j$, and each index has 2 choices in $\pm1$, leaving $4 \cdot 3 = 12$ second neighbors.
The total weight of the 6 first neighbors is $6 \cdot \frac{1}{18} = \frac{1}{3}$.
The total weight of the 12 second neighbors is $12 \cdot \frac{1}{36} = \frac{1}{3}$.

The state of the simulation at a time step is given the \textit{population} of particles, 
$f_p(x,t)$, where the suffix $p$ refers to the discrete velocities above.
In a system with more than one type of particle, each populations of each particle must be maintained separately.  
The movement of populations can be described by the Btatnagar-Gross-Krook update rule:
$$f_p(x + hc_p, t + h) = f_p(x,t) + \omega(x,t) h \left[f_p^{eq}(\rho, \vec{u} - f_p)(x,t) + w_p \frac{c_p \cdot \vec{g}}{c_s^2} \right]$$
Here is a brief description of all the terms appearing in this equation from left to right:
\begin{itemize}
\item $f_p$ is the actual population of the particle introduced above.
\item h is the time step, often taken to be 1 in simplified notation
\item $c_p$ is the displacement vector corresponding a given discrete velocity, e.g. \\
$c_0 = (0,0,0)$,  $c_1 = (1,0,0)$, etc.
\item $\omega$ is the relaxation frequency which is related to the kinematic viscosity $\nu$ (described below)
\item $\rho(x,t)$ is the density of the fluid in this cell, a macroscopic quantity; essentially the 0th moment of the velocity
\item $\vec{u}(x,t)$ is the velocity of the fluid in this cell, a macroscopic quantity; essentially the first moment of the velocity
\item $w_p$ is the weight of particles with each velocity in the D3Q19 scheme; scalars that do not change over the simulation
\item $c_p$ is the discrete velocity
\item $\vec{g}$ is the acceleration applied to the body by external forces; the letter $g$ evokes gravity but it can be any external force
\item $c_s$ is the speed of sound in dimensionless units on this lattice. 
(The speed of sound of the physical medium depends on the gradient of pressure with respect to density).
\end{itemize}

The equilibrium populations can be approximated with a Taylor expansion that
accounts for the 0th, 1st, and 2nd moments of the populations.
$$f_p^{eq}(\rho, \vec{u}) = w_p \rho \left[1 + \frac{\vec{u} \cdot c_p}{c_2^2} + \frac{(\vec{u} \cdot c_p)^2 - c_s^2u^2}{2c_2^4} \right] $$
The resulting $f_p^{eq}$ will match the first two moments (density $\rho$ and velocity $\vec{u}$),
but in general it will $\textbf{not}$ match the second moment (energy density) unelss the system is at equilibrium.
This is how energy dissipation in a viscous fluid away from equilibrium is modeled.
This approximation is valid as long as the system is sufficiently close to equilibrium.
If the parameters are not set properly, the populations can depart from equilibrium by too much
and the simulation will break down.

The macroscopic quantity density $\rho$ is simply the sum of the fluid populations in a cell.
$$\rho(\vec{x}, t) = \sum_{p}f_p(\vec{x}, t)$$
The momentum density is the first moment of the particle velocities.  
This is equal to the density times the velocity in a cell, giving us the formula for $\vec{u}(\vec{x}, t)$:
\begin{align*}
\vec{J}(\vec{x}, t) &= \rho(\vec{x}, t) \vec{u}(\vec{x}, t) = \sum_{p}c_p f_p(\vec{x}, t) \\
\vec{u}(\vec{x}, t) &= \frac{1}{\rho(\vec{x}, t)} \sum_{p}c_p f_p(\vec{x}, t)
\end{align*}

The relaxation frequency is related to the kinematic viscosity $\nu$ by the following relationship:
$$ \nu = c_s^2 \left(\frac{1}{\omega} - \frac{1}{2}\right)$$

A Lattice Boltzmann fluid simulation can be organized into two logical phases, 
which are sometimes called \textit{collision} and \textit{streaming}.
The collision step describes how particles populations in the same cell interact and 
how their weights move toward their equilibrium values as a result of collisions.
The intuition is that when particles hit each other, momentum is conserved, 
but some kinetic energy is dissipated as heat or otherwise.
The equation for the collision step can be written
$$f_p^* = (1-\omega)f_p + \omega f_p^{eq}$$
where $f_p^*$ are called the temporary post-collisional populations.
We can think of these as the new populations one ``instant'' after the previous streaming step.

The streaming step describes how particles of the post-collisional population move forward into the next time step.  
This is very straightforward, the particles just move at their discrete velocities according to
$$f_p(x + c_p, t+ 1) = f_p^*(x, t)$$
A key fact is that streaming is a \textit{local} operation.  
This makes it ideally suited to GPU computations, which excel at simple, highly parallel tasks with memory locality.

Stability is an important concept in the numerical solution of differential equations generally.
One of the best known stability criteria is the 
\href{https://en.wikipedia.org/wiki/Courant%E2%80%93Friedrichs%E2%80%93Lewy_condition}{Courant-Friedrichs-Lewy (CFL) Condition}.
This relates the range of time steps $\Delta t$ for which a numerical method is stable to the spatial discretication $\delta x$,
the speed of movement $u$, and a dimensionless constant that is a property of the numerical method.
A rule of thumb for LBM is that optimal results are achieved when 
$$|u| < \sqrt{\frac{2}{3}} \frac{\Delta x}{\Delta t}$$
A representative value of $u$ is 0.1, leading to a guideline that stable results can be found when the
kinematic viscosity $\nu$ is selected in the range $0.05 < \nu < 1$.

Like any differential equation solution method, LBM must also cope with initial conditions and boundary conditions.
Common initial conditions are prescribed values for the pressure (equivalent to a prescribed density)
and velocity, i.e.
\begin{align*}
\rho(x, t=0) &= \rho_0(x) \\
\vec{u}(x, t=0) &= \vec{u}_0(x)
\end{align*}
A common choice is to set the initial density to be uniform and the initial velocity to be zero.
Once $\rho$ and $\vec{u}$ are initialized, a common modeling choice is to initialize the
populations to the equilibrium implied by these macroscopic variables, i.e.
$$f_p(x, 0) = f_p^{eq}(\rho_0(x), \vec{u}_0(x))$$

Boundary conditions are a bit trickier.  
An astute reader will quickly point out that with these equations as written,
any finite domain will have particles ``falling off the edge'' and without
sensible treatment of boundary conditions, the simulation would just report
that there were no particles left.  That would be sad.
In general, a boundary condition in an LBM simulation can be expressed as 
a linear combination of a constraint on the flux in a direction normal to a wall,
and on the density itself.  
Taking $\phi$ to denote the quantity of interest (can be density or velocity) 
and $n$ the normal direction,
$$b_1 \frac{\partial \phi}{\partial n}(x_b, t) + b_2 \phi(x_b, t) = b_3$$ 

A few special cases are the most common.  
When $\phi = \vec{u} = 0$, we have the \textbf{no slip} boundary condition.
This is common and physically realistic.
When $b1 = 0$, we have a Dirichlet boundary condition, in which the value is imposed.
When $b2 = 0$, we have a Neumann boundary condition, in which the flux is imposed.
When $b1 \ne 0$ and $\b2 \ne 0$, the boundary condition is mixed 
(linear relationship between field value and flux) and is called a Robin boundary condition.
In addition to no-slip boundary conditions, another common choice is for velocity
to be fixed at an \textit{inlet} or \textit{outlet}.
Finally, larger sytems can often be simulated using a \textit{periodic} boundary condition.
For the simulation we ran of an artery, we imposed a non-slip boundary condition on the
side walls of the artery, and a periodic boundary condition on the longitudinal direction,
effectively modeling a longer artery with periodic stenoses.

\section{Description of Code}

\section{Parameters of the Simulation}

\section{Results}

\section{Conclusions and Future Work}

%% bibliography

\begin{thebibliography}{9}
\bibitem{exo} 
Succi, Sauro: The Lattice Boltmann Equation for Fluid Dynamics and Beyond

\end{thebibliography}

\end{document}

