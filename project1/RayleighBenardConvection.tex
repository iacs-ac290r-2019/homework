\documentclass[11pt]{article} % use larger type; default would be 10pt

% Common packages for mathematical articles
\usepackage{fullpage,amsmath,amsfonts,mathpazo,microtype,nicefrac,algorithm2e,graphicx}

% MSE Packages
\usepackage{caption}
\usepackage{physics}
\usepackage{xcolor}
\usepackage{listings}
\usepackage{parskip}

% MSE Make margins a bit smaller
\usepackage[margin=0.75in]{geometry}

% Set-up for hypertext references
\usepackage{hyperref,color,textcomp}
\definecolor{webgreen}{rgb}{0,.35,0}
\definecolor{webbrown}{rgb}{.6,0,0}
\definecolor{RoyalBlue}{rgb}{0,0,0.9}
\hypersetup{
   colorlinks=true, linktocpage=true, pdfstartpage=3, pdfstartview=FitV,
   breaklinks=true, pdfpagemode=UseNone, pageanchor=true, pdfpagemode=UseOutlines,
   plainpages=false, bookmarksnumbered, bookmarksopen=true, bookmarksopenlevel=1,
   hypertexnames=true, pdfhighlight=/O,
   urlcolor=webbrown, linkcolor=RoyalBlue, citecolor=webgreen,
   pdfsubject={Harvard IACS AC 290 R},
   pdfkeywords={},
   pdfcreator={pdfLaTeX},
   pdfproducer={LaTeX with hyperref}
}
\hypersetup{pdftitle={AC290-RBC}}


% MSE Macros
\newcommand{\tty}[1]{\texttt{#1}}
\newcommand{\uu}{\vectorbold{u}}
\newcommand{\vv}{\vectorbold{u}}
\newcommand{\Laplace}{\Delta}

%***Start of Document***
\title{Numerical Simulation of Rayleigh-B\'enard Convection}
\author{Harvard IACS AC 290R - Group 1 \\
Michael S. Emanuel \\
Jonathan Guillotte-Blouin \\
Yue Sun \\
}
\date{14-March-2019} 

\begin{document}
\maketitle

\section{Problem Statement and Motivation}
AC 290R is a course on extreme computing with a focus on the application domain of fluid dynamics.
Fluid dynamics is one of the most mature areas of high performance computation.
Long before there were social networks, scientists and enginners have used 
the most powerful computers available to understand weather systems and aerodynamics.
For our project in this module, we carried out a large scale numerical simulation of 
Rayleigh-B\'enard Convection using the Drekar code running on Harvard's Odyssey supercomputing cluster.  
\href{https://en.wikipedia.org/wiki/Rayleigh%E2\%80%93B%C3%A9nard_convection}
{Rayleigh-B\'enard Convection} (RBC) is a classical problem in fluid dynamics.
It is the phyical phenomenon that arises any time a fluid in a gravitational field has a temperature gradient.
As anyone who lives in an old building can attest, hot air rises and cold air falls.  
This buoyancy effect of warm fluids expanding and becoming less dense is quite general.
When the temperature driven thermal forces are large enough compared to viscous forces,
the fluid will move with characteristic circular flows bringing hot fluid higher and cold fluid lower.
If the the thermal forces become large enough, a turbulent flow will emerge.

A number of scientifically important physical phenomena involve RBC.
Some of the more prominent ones include:
\begin{itemize}
\item\href{https://en.wikipedia.org/wiki/Stellar_evolution}{Stellar evolution}
The gas at the center of a star is hotter than the gas at the outside, and a spherical RBC flow develops
\item\href{https://en.wikipedia.org/wiki/Plate_tectonics}{Plate tectonics}
The center of the earth hot, with a layer of magma on top and a cool crust.  
Over geologic time scales, the earth's mantle behaves like a fluid, and plate tectonics is an instance of RBC.
\item{Weather systems} Warm air rises in the earth's atmosphere and RBC drives weather processes.
\end{itemize}

Our goals in this project are two-fold.  
We would like to learn some basic precepts of fluid dynamics and gain a working knowledge of RBC.
We acknowledge however that fliud dynamics is far too large and complicated a field for us 
to gain a deep understanding in such a short time.
Our primary goal, as suggested by the course title \textbf{Extreme Computing}
is to learn state of the art techniques in scientific computing at a large scale.
We choose to work on a real problem, RBC, rather than a toy problem because
it is both more interesting and we will learn more.

The particular problem we simulated numerically is a two dimensional turbulent Rayleigh-B\'enard Convection.  
This is described mathematically with a set of three equations called the Boussinesq equations.
The \href{https://en.wikipedia.org/wiki/Boussinesq_approximation_(buoyancy)}{Boussinesq approximation} 
is based on a linearization of the buoyancy effect.
The buoyancy effect is approximated as 
$$ \rho(T_0 + \delta T) \approx \rho(T_0) - \frac{\partial \rho}{\partial T}\Bigr|_{T=T_0}$$
We define the coefficient of volume expansion as
\footnote{See here a discussion of  \href{https://en.wikipedia.org/wiki/Thermal_expansion}{thermal expansion} generally}
$$\alpha_V = -\frac{1}{\rho_0}\frac{\partial \rho}{\partial T}\Bigr|_{T=T_0}$$
The coefficient of volume expansion is a property of a fluid.
The second simplifying assumption made in the Boussinesq equations is that fluids are incompressible
(except for this linear sensitivity of their density to temperature).
The Navier-Stokes equations then simplify to the three Boussinesq equations:
\begin{align}
\frac{\partial \uu}{\partial t} + \grad \cdot (\uu \otimes \uu) &= \frac{1}{\rho_0} \grad P + \nu \grad^2 \uu + \alpha_V g T \hat{y} \\
\grad \cdot \uu &= 0 \\
\frac{\partial T}{\partial t} + \grad \cdot (uT) &= \kappa \grad^2 T
\end{align}

In these equations, $\rho_0$ is is the density at the reference temperature $T_0$.
$\nu$ is the fluid viscosity and $\kappa$ is the thermal diffusivity.

The presentation above includes the physical constants and is suitable for engineering.
For mathematics, it is often preferable to use nondimensional equations.
\footnote{See this article on the technique of \href{https://en.wikipedia.org/wiki/Nondimensionalization}{nomdimensionalization}.}
There are three choices that have been used for nondimensionalizing the Boussinesq equations.
We follow the classical approach and choose the following dimensionless parameters:
\begin{itemize}
\item{$\Delta T = T_{bot} - T_{top}$} the temperature difference; this is the temperature scale
\item{$H=y_{top} - y_{bot}$} the height of the channel; this is the length scale
\item{$\tau = \nicefrac{H^2}{\kappa}$} is the time scale; this choice is called ``thermal scaling''
\item{$U = \nicefrac{H}{\tau}$} is the velocity scale
\item{$\rho_0 U^2$} is the pressure scale
\end{itemize}

With this choice of dimensionless parameters, we have the non-dimensional Boussinequ equations:
\begin{align}
\frac{\partial \uu}{\partial t} + \grad \cdot (\uu \otimes \uu) &= \grad P + \textrm{Pr} \grad^2 \uu + \textrm{RaPr}T \hat{y} \\
\grad \cdot \uu &= 0 \\
\frac{\partial T}{\partial t} + \grad \cdot (uT) &=  \grad^2 T
\end{align}

There are two dimensionless ratios appearing:
\begin{itemize}
\item{Pr} is the \href{https://en.wikipedia.org/wiki/Prandtl_number}{Prandtl number}, $\textrm{Pr} = \nicefrac{\nu}{\kappa}$.  
This is the ratio of the thermal viscosity to the thermal diffusivity and is a property of the fluid.
\item{Ra} is the famous \href{https://en.wikipedia.org/wiki/Rayleigh_number}{Rayleigh number}.  It is defined by
$$\textrm{Ra} = \frac{\alpha_V \Laplace T g H^3}{\nu \kappa}$$
The Rayleigh number describes the ratio of temperature driven forces to viscous forces.
For this reason, a very low Rayleigh number leads to no convective flow; 
an intermediate Rayleigh number leads to a somewhat regular flow;
and a high Rayleigh number leads to a turbulent flow.
\end{itemize}

\section{Description of Code}

\section{Overview of Numerical Methods Used}

\section{Parameters of the Simulation}

\section{Results}

\section{Conclusions and Future Work}

Your text goes here.

\subsection{A subsection}

More text.

\end{document}
